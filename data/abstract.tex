% !TeX root = ../thuthesis-example.tex

% 中英文摘要和关键字

\begin{abstract}

随着物联网技术的飞速发展,时序数据库在存储和管理海量设备数据方面扮演着至关重要的角色。Apache IoTDB 作为一款流行的开源时序数据库,其在分布式部署场景下的高可用性和容错能力对于保障工业物联网应用服务连续性、数据安全性至关重要。

在面临节点宕机、复杂网络非对称分区等故障时,Apache IoTDB会出现服务长时间不可用、性能退化、数据丢失等问题。
针对现有问题,系统性地梳理总结分布式数据库的普遍故障场景的特性和共性,提出并实现一套针对 Apache IoTDB 的高可用与容错框架,通过系统各组件的协同,做到两个关键指标恢复点目标(RPO)为零、恢复时间目标(RTO)为分钟级的容错能力,是本文的研究核心。

本文围绕着Apache IoTDB的高可用和容错能力展开研究,主要贡献如下:

1. 对分布式时序数据库在运行中潜在的故障进行系统性的梳理、分类和分析。本文将种类繁多的故障总结为节点宕机、磁盘资源不足、网络分区(对称分区和非对称分区)、集群状态短暂不一致五类,分析每一种故障的原因、对系统的影响,并基于这些故障进行后续框架的设计和实验的验证。

2. 提出针对 Apache IoTDB 系统的高可用与容错框架。该框架依赖共识协议提供的基础能力,划分为故障检测、故障恢复、\failover 三部分。框架明确系统各组件,即客户端(Session)、管理节点(ConfigNode)、数据节点(DataNode)之间的协同机制。

3. 上述工作在 Apache IoTDB 系统中进行实现,并在测试环境中对框架能力做全面测试与评估。针对1中提出的五类故障场景,系统实现的故障容错机制能够有效工作,并实现 RPO 指标为零、RTO 指标为分钟级的效果。

  % 关键词用“英文逗号”分隔,输出时会自动处理为正确的分隔符
  \thusetup{
    keywords = {工业物联网, 时序数据库, 高可用和容错, 故障检测和故障转移, RTO和RPO},
  }
\end{abstract}

\begin{abstract*}
  
  With the rapid development of Internet of Things technology, time series databases play a crucial role in storing and managing massive amounts of device data. As a popular open-source time series database, the high availability and fault tolerance capabilities of Apache IoTDB in distributed deployment scenarios are essential for ensuring the continuity of industrial IoT application services and data security.

  Apache IoTDB distributed time series database has potential areas for optimization when facing failures such as process crash and complex asymmetric network partitions, where the Recovery Time Objective (RTO) is in the order of hours and the Recovery Point Objective (RPO) is non-zero. How to systematically summarize common failure scenarios in actual operation, propose and implement a high availability and fault tolerance framework specifically for Apache IoTDB, and achieve zero RPO and minute-level RTO fault tolerance capabilities through the collaboration of various system components is the key problem studied in this paper.

  This paper focuses on the high availability and fault tolerance capabilities of Apache IoTDB, with the following main contributions:

  1. A systematic review, classification, and analysis of potential failures in distributed time series databases during operation. This paper summarizes the wide variety of failures into five categories: node downtime, insufficient disk resources, network partitions (symmetric and asymmetric), and transient cluster state inconsistency, and analyzes the cause and impact of each failure on the system. This work serves as the fundamental basis for the subsequent design of the high availability and fault tolerance framework.
  
  2. A high availability and fault tolerance framework for the Apache IoTDB system is proposed. This framework relies on the basic capabilities provided by consensus protocols and is divided into three parts: fault detection, fault recovery, and failover. The framework clarifies the collaboration mechanism among various system components, namely the Session, ConfigNode, and DataNode.
  
  3. The aforementioned work has been implemented in the Apache IoTDB system and fully tested and evaluated for framework capabilities in a test environment. For the five types of failure scenarios proposed in point 1, the implemented fault tolerance mechanism effectively works and achieves results with an RPO of zero and an RTO on the order of minutes.


  % Use comma as separator when inputting
  \thusetup{
    keywords* = {Industrial Internet of Things, Time-series Database, High Availability and Fault Tolerance, Failure Detection and Failover, RTO/RPO},
  }
\end{abstract*}
