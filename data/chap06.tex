% !TeX root = ../thuthesis-example.tex

\chapter{实验章节}

\section{实验准备}

\subsection{测试环境}

\section{故障检测和判断}

\subsection{Phi Accural故障检测器的行为表现}
1.1 Phi 检测器 对于 Elasped Time 的行为

1.2 Phi 对于 Pause 接受度的行为
\subsection{Phi Accural相较于Fixed的对比}
1.3 Phi 相较于 Fixed 在发现错误的平均延迟

1.4 Phi 相较于 环境的动态适应行为。

\subsection{非对称网络分区的检测速度}

2.1 非对称网络分区之后的检测速度
2.2 非对称网络分区恢复后的检测速度

\subsection{基于Thrift连接的进程宕机检测速度}

基于TCP连接的问题检测行为

\section{副本和共识协议}


故障处理:

0. 不同共识协议的故障表现(这个感觉跟你的毕设关系不大?新豪之前做过一版,可以跟他聊聊)

0.1 主副本失效之后的failover时间
    2副本 3副本 5副本 三个不同共识协议的比较
0.2 副本回归之后的catch up恢复时间
    三个不同共识协议的比较

2.  非对称网络分区的处理效果




端到端测试:

4. 不同故障场景不同的RTO情况
SR 统一用 Ratis,DataRegion 可以换 RatisConsensus, IoTConsensus, IoTConsensus
4.1 磁盘写满(用 readonly 就行)
4.2 进程宕机(测试 kill -9 和脚本停止两种场景)
4.3 非对称网络分区(A,B,C 三个节点,其中两个节点之间无法通信)
4.4 对称网络分区(A,B,C 三个节点,其中 A 与其它两个节点无法通信)
4.5 集群变更(并发读写过程中测试缩容)

5. 性能损失
5.1 规划的性能损失
5.2 调度的性能损失
5.3 benchmark端到端在进程宕机、网络分区、非对称网络分区的吞吐



\section{Phi的实验章节}\label{exp-phi}
