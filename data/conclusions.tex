% !TeX root = ../thuthesis-example.tex

\chapter{总结与展望}


\section{本文总结}

本文从工业物联网时序数据库Apache IoTDB的高可用和容错能力出发,在梳理Apache IoTDB现有的高可用建设、调研了主流的分布式数据库对高可用性和容错能力的建设滞后,从整体框架、故障检测、故障转移和恢复、副本和共识等角度给出了基于Apache IoTDB的高可用和容错性的优化,并在最后通过相应的实验对本文的工作进行了有效性的验证和综合的评估。
本文的主要贡献如下:

1. 为Apache IoTDB设计了整体的高可用和容错框架。本文并非仅仅针对某个局部问题进行优化,而是从全局视角出发,系统性地设计了一套完整的高可用和容错架构。该框架包括了使用副本复制和共识协议作为容错基础、使用故障检测和研判算法作为前提、使用错误自动转移和恢复能力作为应对手段,综合IoTDB的各个子系统协同工作。这个框架为后续的具体优化措施提供了理论基础和指导方向。


2. 优化了Apache IoTDB的故障检测能力,覆盖了更全面的故障种类,也在部分故障情况下加快了故障发现速度。同时通过引入新的故障研判算法来提高研判的准确性、降低误报率。本文的工作能够顺利覆盖非对称网络分区等少见问题,并且针对进程宕机失效问题进行了优化,且通过引入基于Phi Accrual的故障检测算法替换了机械的心跳超时算法。


3. 优化了Apache IoTDB的故障转移和容错能力。本文通过建设ConfigNode、Session、协调者共识模块的协同,在请求阶段、规划阶段和共识阶段共同的配合下,使用多副本来进行故障转移,最终实现Apache IoTDB在不同故障场景下的高可用能力。

4. 进行了大量的实验,针对常见的故障评估了 Apache IoTDB的RTO和RPO表现。本文给出了对故障检测能力和共识协议特性的研究,并且针对磁盘写满、进程宕机、对称网路分区、非对称网络分区、集群变更场景下的RPO和RTO情况测试,验证了Apache IoTDB在这些故障场景下的可用性。


\section{未来工作展望}

本文设计并实现的Apache IoTDB的高可用和容错能力虽然取得了一定的成果,但在实际应用中仍然可能面临各种复杂和严峻的挑战,因此还存在多个值得进一步改进和完善之处,主要可以细分为以下几个方面。

首先,是支持更多极端场景下的故障容错和高可用能力。 目前的优化方案可能主要关注了较为常见的故障类型,例如单节点宕机、简单的网络中断等。然而,在真实的工业物联网环境中,可能会出现更加复杂的极端故障场景,例如:

1. 多节点同时失效: 在大规模集群中,可能会由于电力故障、机房事故、或者人为误操作等原因导致多个节点在短时间内同时发生故障。当前的容错机制可能需要在这种情况下进一步增强其韧性,确保数据和服务不会完全中断。

2. 数据中心级别的故障: 对于一些关键应用,可能需要考虑整个数据中心级别的故障,例如自然灾害等。这需要引入跨数据中心的复制和容灾方案。

3. 存储介质的彻底损坏: 除了节点宕机,单个节点的硬盘和内存等存储介质也可能发生彻底损坏或者部分损坏,导致数据永久丢失或部分变脏。需要更完善的数据恢复策略来应对这种情况。

其次,是配合主动式预防容错基础,进一步提高集群的可用性。目前的容错机制可能更多的是一种被动响应式的策略,即在故障发生后才进行恢复和转移。为了进一步提升系统的可靠性和可用性,可以考虑引入主动式预防容错机制。例如基于预测的Region迁移: 通过监控节点的资源使用情况(例如CPU、内存、磁盘IO等)和健康状态,预测潜在的故障风险。在风险较高时,可以主动将该节点上的数据主副本迁移到其他健康节点,从而避免在故障真正发生时才进行被动的转移,减少对服务的影响。

最后,是为不同需求的用户提供不同级别的高可用能力。不同的工业物联网应用场景对高可用和容错的需求可能不同。一些场景可能对数据的强一致性要求非常高,而另一些场景可能更关注系统的可用性,允许一定程度的数据延迟或不一致。因此,提供用户可选的高可用配置将能够更好地满足不同用户的需求。例如:

1. 不同的故障转移策略:用户可以选择自动故障转移还是手动故障转移。自动故障转移可以更快地恢复服务,但可能存在误判的风险;手动故障转移则需要人工干预,但可以更谨慎地进行操作。

2. 不同的重试和等待策略:用户可以选择更长的重试,进一步提高请求成功的概率,或是更少的重试,防止重试策略对系统的时延产生影响。