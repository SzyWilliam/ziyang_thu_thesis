% !TeX root = ../thuthesis-example.tex

\chapter{总结与展望}


\section{本文总结}

本文从工业物联网时序数据库Apache IoTDB的高可用和容错能力出发,在梳理Apache IoTDB现有的高可用建设、调研了主流的分布式数据库对高可用性和容错能力的建设滞后,对分布式时序数据库的常见问题进行分类和梳理,提出了Apache IoTDB的高可用和容错能力的普遍框架,
从整体框架、故障检测、\failover 和恢复、副本和共识等角度给出了基于Apache IoTDB的高可用和容错性的优化,并在最后通过相应的实验对本文的工作进行了有效性的验证和综合的评估。


\section{未来工作展望}

本文设计并实现的Apache IoTDB的高可用和容错能力虽然取得了一定的成果,但在实际应用中仍然可能面临各种复杂和严峻的挑战,因此还存在多个值得进一步改进和完善之处,主要可以细分为以下几个方面。

首先,是支持更多极端场景下的故障容错和高可用能力。 目前的优化方案可能主要关注了较为常见的故障类型,例如单节点宕机、简单的网络中断等。然而,在真实的工业物联网环境中,可能会出现更加复杂的极端故障场景,例如:

1. 多节点同时失效: 在大规模集群中,可能会由于电力故障、机房事故、或者人为误操作等原因导致多个节点在短时间内同时发生故障。当前的容错机制可能需要在这种情况下进一步增强其韧性,确保数据和服务不会完全中断。

2. 数据中心级别的故障: 对于一些关键应用,可能需要考虑整个数据中心级别的故障,例如自然灾害等。这需要引入跨数据中心的复制和容灾方案。

3. 存储介质的彻底损坏: 除了节点宕机,单个节点的硬盘和内存等存储介质也可能发生彻底损坏或者部分损坏,导致数据永久丢失或部分变脏。需要更完善的数据恢复策略来应对这种情况。

其次,是配合主动式预防容错基础,进一步提高集群的可用性。目前的容错机制可能更多的是一种被动响应式的策略,即在故障发生后才进行恢复和转移。为了进一步提升系统的可靠性和可用性,可以考虑引入主动式预防容错机制。例如基于预测的分区迁移: 通过监控节点的资源使用情况(例如CPU、内存、磁盘IO等)和健康状态,预测潜在的故障风险。在风险较高时,可以主动将该节点上的数据主副本迁移到其他健康节点,从而避免在故障真正发生时才进行被动的转移,减少对服务的影响。

最后,是为不同需求的用户提供不同级别的高可用能力。不同的工业物联网应用场景对高可用和容错的需求可能不同。一些场景可能对数据的强一致性要求非常高,而另一些场景可能更关注系统的可用性,允许一定程度的数据延迟或不一致。因此,提供用户可选的高可用配置将能够更好地满足不同用户的需求。例如:

1. 不同的\failover 策略:用户可以选择自动\failover 还是手动\failover 。自动\failover 可以更快地恢复服务,但可能存在误判的风险;手动\failover 则需要人工干预,但可以更谨慎地进行操作。

2. 不同的重试和等待策略:用户可以选择更长的重试,进一步提高请求成功的概率,或是更少的重试,防止重试策略对系统的时延产生影响。